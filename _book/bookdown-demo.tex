\documentclass[]{book}
\usepackage{lmodern}
\usepackage{amssymb,amsmath}
\usepackage{ifxetex,ifluatex}
\usepackage{fixltx2e} % provides \textsubscript
\ifnum 0\ifxetex 1\fi\ifluatex 1\fi=0 % if pdftex
  \usepackage[T1]{fontenc}
  \usepackage[utf8]{inputenc}
\else % if luatex or xelatex
  \ifxetex
    \usepackage{mathspec}
  \else
    \usepackage{fontspec}
  \fi
  \defaultfontfeatures{Ligatures=TeX,Scale=MatchLowercase}
\fi
% use upquote if available, for straight quotes in verbatim environments
\IfFileExists{upquote.sty}{\usepackage{upquote}}{}
% use microtype if available
\IfFileExists{microtype.sty}{%
\usepackage{microtype}
\UseMicrotypeSet[protrusion]{basicmath} % disable protrusion for tt fonts
}{}
\usepackage[margin=1in]{geometry}
\usepackage{hyperref}
\hypersetup{unicode=true,
            pdftitle={A Minimal Book Example},
            pdfauthor={Yihui Xie},
            pdfborder={0 0 0},
            breaklinks=true}
\urlstyle{same}  % don't use monospace font for urls
\usepackage{natbib}
\bibliographystyle{apalike}
\usepackage{color}
\usepackage{fancyvrb}
\newcommand{\VerbBar}{|}
\newcommand{\VERB}{\Verb[commandchars=\\\{\}]}
\DefineVerbatimEnvironment{Highlighting}{Verbatim}{commandchars=\\\{\}}
% Add ',fontsize=\small' for more characters per line
\usepackage{framed}
\definecolor{shadecolor}{RGB}{248,248,248}
\newenvironment{Shaded}{\begin{snugshade}}{\end{snugshade}}
\newcommand{\AlertTok}[1]{\textcolor[rgb]{0.94,0.16,0.16}{#1}}
\newcommand{\AnnotationTok}[1]{\textcolor[rgb]{0.56,0.35,0.01}{\textbf{\textit{#1}}}}
\newcommand{\AttributeTok}[1]{\textcolor[rgb]{0.77,0.63,0.00}{#1}}
\newcommand{\BaseNTok}[1]{\textcolor[rgb]{0.00,0.00,0.81}{#1}}
\newcommand{\BuiltInTok}[1]{#1}
\newcommand{\CharTok}[1]{\textcolor[rgb]{0.31,0.60,0.02}{#1}}
\newcommand{\CommentTok}[1]{\textcolor[rgb]{0.56,0.35,0.01}{\textit{#1}}}
\newcommand{\CommentVarTok}[1]{\textcolor[rgb]{0.56,0.35,0.01}{\textbf{\textit{#1}}}}
\newcommand{\ConstantTok}[1]{\textcolor[rgb]{0.00,0.00,0.00}{#1}}
\newcommand{\ControlFlowTok}[1]{\textcolor[rgb]{0.13,0.29,0.53}{\textbf{#1}}}
\newcommand{\DataTypeTok}[1]{\textcolor[rgb]{0.13,0.29,0.53}{#1}}
\newcommand{\DecValTok}[1]{\textcolor[rgb]{0.00,0.00,0.81}{#1}}
\newcommand{\DocumentationTok}[1]{\textcolor[rgb]{0.56,0.35,0.01}{\textbf{\textit{#1}}}}
\newcommand{\ErrorTok}[1]{\textcolor[rgb]{0.64,0.00,0.00}{\textbf{#1}}}
\newcommand{\ExtensionTok}[1]{#1}
\newcommand{\FloatTok}[1]{\textcolor[rgb]{0.00,0.00,0.81}{#1}}
\newcommand{\FunctionTok}[1]{\textcolor[rgb]{0.00,0.00,0.00}{#1}}
\newcommand{\ImportTok}[1]{#1}
\newcommand{\InformationTok}[1]{\textcolor[rgb]{0.56,0.35,0.01}{\textbf{\textit{#1}}}}
\newcommand{\KeywordTok}[1]{\textcolor[rgb]{0.13,0.29,0.53}{\textbf{#1}}}
\newcommand{\NormalTok}[1]{#1}
\newcommand{\OperatorTok}[1]{\textcolor[rgb]{0.81,0.36,0.00}{\textbf{#1}}}
\newcommand{\OtherTok}[1]{\textcolor[rgb]{0.56,0.35,0.01}{#1}}
\newcommand{\PreprocessorTok}[1]{\textcolor[rgb]{0.56,0.35,0.01}{\textit{#1}}}
\newcommand{\RegionMarkerTok}[1]{#1}
\newcommand{\SpecialCharTok}[1]{\textcolor[rgb]{0.00,0.00,0.00}{#1}}
\newcommand{\SpecialStringTok}[1]{\textcolor[rgb]{0.31,0.60,0.02}{#1}}
\newcommand{\StringTok}[1]{\textcolor[rgb]{0.31,0.60,0.02}{#1}}
\newcommand{\VariableTok}[1]{\textcolor[rgb]{0.00,0.00,0.00}{#1}}
\newcommand{\VerbatimStringTok}[1]{\textcolor[rgb]{0.31,0.60,0.02}{#1}}
\newcommand{\WarningTok}[1]{\textcolor[rgb]{0.56,0.35,0.01}{\textbf{\textit{#1}}}}
\usepackage{longtable,booktabs}
\usepackage{graphicx,grffile}
\makeatletter
\def\maxwidth{\ifdim\Gin@nat@width>\linewidth\linewidth\else\Gin@nat@width\fi}
\def\maxheight{\ifdim\Gin@nat@height>\textheight\textheight\else\Gin@nat@height\fi}
\makeatother
% Scale images if necessary, so that they will not overflow the page
% margins by default, and it is still possible to overwrite the defaults
% using explicit options in \includegraphics[width, height, ...]{}
\setkeys{Gin}{width=\maxwidth,height=\maxheight,keepaspectratio}
\IfFileExists{parskip.sty}{%
\usepackage{parskip}
}{% else
\setlength{\parindent}{0pt}
\setlength{\parskip}{6pt plus 2pt minus 1pt}
}
\setlength{\emergencystretch}{3em}  % prevent overfull lines
\providecommand{\tightlist}{%
  \setlength{\itemsep}{0pt}\setlength{\parskip}{0pt}}
\setcounter{secnumdepth}{5}
% Redefines (sub)paragraphs to behave more like sections
\ifx\paragraph\undefined\else
\let\oldparagraph\paragraph
\renewcommand{\paragraph}[1]{\oldparagraph{#1}\mbox{}}
\fi
\ifx\subparagraph\undefined\else
\let\oldsubparagraph\subparagraph
\renewcommand{\subparagraph}[1]{\oldsubparagraph{#1}\mbox{}}
\fi

%%% Use protect on footnotes to avoid problems with footnotes in titles
\let\rmarkdownfootnote\footnote%
\def\footnote{\protect\rmarkdownfootnote}

%%% Change title format to be more compact
\usepackage{titling}

% Create subtitle command for use in maketitle
\newcommand{\subtitle}[1]{
  \posttitle{
    \begin{center}\large#1\end{center}
    }
}

\setlength{\droptitle}{-2em}
  \title{A Minimal Book Example}
  \pretitle{\vspace{\droptitle}\centering\huge}
  \posttitle{\par}
  \author{Yihui Xie}
  \preauthor{\centering\large\emph}
  \postauthor{\par}
  \predate{\centering\large\emph}
  \postdate{\par}
  \date{2018-11-30}

\usepackage{booktabs}
\usepackage{amsthm}
\makeatletter
\def\thm@space@setup{%
  \thm@preskip=8pt plus 2pt minus 4pt
  \thm@postskip=\thm@preskip
}
\makeatother

\usepackage{amsthm}
\newtheorem{theorem}{Theorem}[chapter]
\newtheorem{lemma}{Lemma}[chapter]
\theoremstyle{definition}
\newtheorem{definition}{Definition}[chapter]
\newtheorem{corollary}{Corollary}[chapter]
\newtheorem{proposition}{Proposition}[chapter]
\theoremstyle{definition}
\newtheorem{example}{Example}[chapter]
\theoremstyle{definition}
\newtheorem{exercise}{Exercise}[chapter]
\theoremstyle{remark}
\newtheorem*{remark}{Remark}
\newtheorem*{solution}{Solution}
\begin{document}
\maketitle

{
\setcounter{tocdepth}{1}
\tableofcontents
}
\hypertarget{prerequisites}{%
\chapter{Prerequisites}\label{prerequisites}}

This is a \emph{sample} book written in \textbf{Markdown}. You can use
anything that Pandoc's Markdown supports, e.g., a math equation
\(a^2 + b^2 = c^2\).

The \textbf{bookdown} package can be installed from CRAN or Github:

\begin{Shaded}
\begin{Highlighting}[]
\KeywordTok{install.packages}\NormalTok{(}\StringTok{"bookdown"}\NormalTok{)}
\CommentTok{# or the development version}
\CommentTok{# devtools::install_github("rstudio/bookdown")}
\end{Highlighting}
\end{Shaded}

Remember each Rmd file contains one and only one chapter, and a chapter
is defined by the first-level heading \texttt{\#}.

To compile this example to PDF, you need XeLaTeX. You are recommended to
install TinyTeX (which includes XeLaTeX):
\url{https://yihui.name/tinytex/}.

\hypertarget{values}{%
\chapter{Values}\label{values}}

Here are the lab values

\hypertarget{about-the-lab}{%
\chapter{About the lab}\label{about-the-lab}}

\hypertarget{our-research}{%
\section{Our research}\label{our-research}}

The lab generally examines how we remember things; how we make
decisions; and how we plan. Some examples of the types of questions we
address are:

\begin{itemize}
\tightlist
\item
  How do we encode and recall sequential information?
\item
  How is memory affected by reward, or instructions to prioritise
  information?
\item
  How do we sample information from memory to remember, form judgements,
  and make choices?
\item
  How can we use evidence accumulation models to understand
  decision-making?
\item
  How do we make sequences of inter-related decisions, particularly when
  pursuing multiple goals?
\item
  How are our judgements and choices affected by others?
\item
  How do small groups collate information and make judgements?
\end{itemize}

\hypertarget{lab-structure}{%
\section{Lab structure}\label{lab-structure}}

Different people in the lab are at different stages in their research
careers. A list of current members, collaborators and alumni can be
found \href{link}{here}

\hypertarget{academics}{%
\subsection{Academics}\label{academics}}

Simon Farrell is the head of the lab. You will usually be working
directly with Simon in carrying out your research.

We are also part of a larger lab group called the Cognitive Science
Group. The other academics in this group are Ullrich Ecker and Mark
Hurlstone.

\hypertarget{post-docs}{%
\subsection{Post-docs}\label{post-docs}}

Postdoctoral researchers (or postdocs) are people who have finished
their PhD, and are (usually) employed full-time on a research project.

\hypertarget{phd-students}{%
\subsection{PhD students}\label{phd-students}}

PhD students are carrying out a research project (typically over the
course of 3--4 years) that will ultimately lead to a PhD. A PhD
typically involves running a number of experiments. Some of our PhD
students on a combined programme, meaning that they are completing a
research project in parallel with a Masters programme (typically over 4
years).

\hypertarget{honours-students}{%
\subsection{Honours students}\label{honours-students}}

Honours students work on a small research project (typically 1--2
experiments) over 8--9 months as part of their Honours degree.

\hypertarget{external-collaborators}{%
\subsection{External collaborators}\label{external-collaborators}}

We have a number of external collaborators: researchers from other
universities around the world who bring their own views and expertise to
projects. Working with others is fun!

\hypertarget{communication}{%
\chapter{Communication}\label{communication}}

\hypertarget{lab-meetings}{%
\section{Lab meetings}\label{lab-meetings}}

Lab meetings are a primary means of catching up, setting context,
sharing results, getting feedback, making decisions, and planning. We
have lab meetings on average every fortnight, but the exact schedule is
determined on a week by week basis. \textbf{Attendance at lab meetings
is expected}; however, we know that it is difficult to organise a time
that suits everyone, and other factors may affect your ability to attend
individual meetings.

The main structure of the lab meeting is:

\begin{enumerate}
\def\labelenumi{\arabic{enumi}.}
\tightlist
\item
  Each lab member gives a brief (1--2 min) update on where they are at.
  The purpose is to check in and make sure everyone is OK, and identify
  any road blocks that will need to be addressed outside the meeting.
\item
  Any lab members that have results to present do so. Often this will
  just be a casual presentation, so only a brief outline of background
  and method is needed, and discussion should focus on the results and
  their interpretation (the science). However, lab members may also
  practice a presentation, in which case they may also welcome feedback
  on the presentation itself.
\item
  Sometimes, we will have discussion of lab issues affecting the group
  (e.g., in the past we have discussed the format of project
  directories, norms for pre-registation, and the lab logo).
\end{enumerate}

So that the lab meetings run efficiently and are enjoyable, there are
some expectations of all members:

\begin{enumerate}
\def\labelenumi{\arabic{enumi}.}
\item
  Everyone takes part in lab discussion; even if something is not in
  your area, you should be able to say something about it (or say why
  you can't say something about it). This is to have you practice giving
  feedback, and to clarify your own understanding of different areas.
\item
  A corollary to 1. is: give other lab members room to talk, and try not
  to dominate the conversation.
\item
  A good heuristic is to follow the \href{link}{Griceam maxims}
\end{enumerate}

\begin{enumerate}
\def\labelenumi{\alph{enumi}.}
\tightlist
\item
  The maxim of quantity, where one tries to be as informative as one
  possibly can, and gives as much information as is needed, and no more.
\item
  The maxim of quality, where one tries to be truthful, and does not
  give information that is false or that is not supported by evidence.
\item
  The maxim of relation, where one tries to be relevant, and says things
  that are pertinent to the discussion. The maxim of manner, when one
  tries to be as clear, as brief, and as orderly as one can in what one
  says, and where one avoids obscurity and ambiguity.
\end{enumerate}

What does this mean in practice? Think before speaking, don't dominate
the conversation (see 2. above), and make immediately useful suggestions
(you can always suggest getting coffee afterwards if you have some more
expansive and philosophical points to make)

If you are presenting, some of the discussion might be in the form of
feedback. Remember, feedback is not criticism, and should not be taken
as such. Indeed, it would be good to get into the practice of asking for
feedback, as you can only learn from feedback.

\hypertarget{slack}{%
\section{Slack}\label{slack}}

We use Slack for chats and conversations. An advantage is that we can
easily include future collaborators (including future students) and they
easily have access to history of the project. \emph{Warning}: there are
no guarantess about the search/archiving facilities of Slack. Slack
should notnserve as a permanent record of scientific (or other)
discussion. (Note, though, that we do use it to keep some pinned info).
Summaries of discussions (especially important for remembering and
understanding why we made the decisions we did, which we will need to
justify in papers) should go into google docs/dropbox/whatever.

\hypertarget{email}{%
\section{Email}\label{email}}

Use email primarily for correspondence of a more formal nature. It's
much easier to Slack than to email, so please use Slack in first
instance unless business is of formal nature (e.g., relating to
enrolment/formal issues), or you are forwarding along some other info.

\hypertarget{google-docs}{%
\section{Google docs}\label{google-docs}}

You should be using something like google docs or dropbox (or even a
github wiki) as a record of meetings and decisions made. This is where
you might also make notes of conversations from Slack (even just cut and
paste important convos that you can't otherwise incorporate as comments
on papers or put into effect immediately).

\hypertarget{github-wiki}{%
\section{Github wiki}\label{github-wiki}}

Once folks are up to scratch with git + GitHub, we will explore using
issues on GitHub as a way of managing specific tasks. It looks like it
might be quite unwieldy, but is used by other labs to track things.

\hypertarget{honours-students-1}{%
\chapter{Honours students}\label{honours-students-1}}

This chapter will be advice specific to Honours students

\hypertarget{phd-students-1}{%
\chapter{PhD students}\label{phd-students-1}}

This chapter will be advice specific to PhD students

\bibliography{book.bib,packages.bib}


\end{document}
